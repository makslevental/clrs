%% LyX 2.0.8.1 created this file.  For more info, see http://www.lyx.org/.
%% Do not edit unless you really know what you are doing.
\documentclass[oneside,english]{amsart}
\usepackage[T1]{fontenc}
\usepackage[latin9]{inputenc}
\usepackage{listings}
\usepackage{float}
\usepackage{amsthm}
\usepackage{amssymb}
\PassOptionsToPackage{normalem}{ulem}
\usepackage{ulem}

\makeatletter

%%%%%%%%%%%%%%%%%%%%%%%%%%%%%% LyX specific LaTeX commands.
\floatstyle{ruled}
\newfloat{algorithm}{tbp}{loa}
\providecommand{\algorithmname}{Algorithm}
\floatname{algorithm}{\protect\algorithmname}

%%%%%%%%%%%%%%%%%%%%%%%%%%%%%% Textclass specific LaTeX commands.
\numberwithin{equation}{section}
\numberwithin{figure}{section}
\theoremstyle{plain}
\newtheorem{thm}{\protect\theoremname}
  \theoremstyle{plain}
  \newtheorem{lem}[thm]{\protect\lemmaname}
  \theoremstyle{plain}
  \newtheorem{cor}[thm]{\protect\corollaryname}
  \theoremstyle{plain}
  \newtheorem*{thm*}{\protect\theoremname}
  \theoremstyle{definition}
  \newtheorem{xca}[thm]{\protect\exercisename}
  \theoremstyle{definition}
  \newtheorem{problem}[thm]{\protect\problemname}

\makeatother

\usepackage{babel}
  \providecommand{\corollaryname}{Corollary}
  \providecommand{\exercisename}{Exercise}
  \providecommand{\lemmaname}{Lemma}
  \providecommand{\problemname}{Problem}
  \providecommand{\theoremname}{Theorem}
\providecommand{\theoremname}{Theorem}

\begin{document}

\title{\textup{Proofs of shortest-paths properties}}
\maketitle
\begin{lem}
\textbf{\textup{\label{lem:Subpaths-of-shortest}Subpaths of shortest
paths are shortest paths.}}\end{lem}
\begin{thm}
\textbf{\textup{Triangle inequality}}\textup{. Let $G=\left(V,E\right)$
be a weighted, directed graph with weight function $w\,:\, E\rightarrow\mathbb{R}$
and source vertex $s$. Then for all edges $\left(u,v\right)\in E$,
we have $\delta\left(s,v\right)\leq\delta\left(s,u\right)+w\left(\left(u,v\right)\right)$}\end{thm}
\begin{proof}
Let $p=\left\langle s,\dots,v\right\rangle $ be the shortest path
from $s$ to $v$. Then $p$ has length $\delta\left(s,v\right)$
and since $p$ is the shortes path $\delta\left(s,v\right)$ is less
than or equal to the length of any other path from $s$ to $v$. In
particular the path that includes the shortest path from $s$ to $u$
plus the edge from $u$ to $v$. If there is no path from $s$ to
$v$ then define $\delta\left(s,v\right)\equiv-\infty$.\end{proof}
\begin{lem}
\textbf{\textup{Upper-bound property}}\textup{. Let $G=\left(V,E\right)$
be a weighted, directed graph with weight function $w\,:\, E\rightarrow\mathbb{R}$
and source vertex $s$ and initialize the graph with }
\begin{algorithm}[H]
\begin{lstlisting}[language=Python,mathescape=true,numbers=left,tabsize=4]
Initialize-Single-Source$\left(G,s\right)$
for each $v\in V$
	$v.d = \infty$
	$v.\pi = $NIL
$s.d = 0$
	
\end{lstlisting}


\caption{Initialize-Single-Source}
\end{algorithm}
\textup{then $v.d\geq\delta\left(s,v\right)$ for all $v\in V$ and
this invariant is maintained over any sequence of relaxation steps
on the edges of $G$. Moreover, once $v.d$ achieves its lower bound
$\delta\left(s,v\right)$, it never changes.}\end{lem}
\begin{proof}
Proof by induction on the number of relaxation steps: 
\begin{algorithm}[H]
\begin{lstlisting}[language=Python,mathescape=true,numbers=left,tabsize=4]
Relax$\left(u,v,w\right)$
if $v.d > u.d + w\left(\left(u,v\right)\right)$
	$v.d = u.d + w\left(\left(u,v\right)\right)$
	$v.\pi = u$
	
\end{lstlisting}


\caption{Relaxation}
\end{algorithm}
For the base case $v.d\geq\delta\left(s,v\right)$ for all $v\in V$
after the initialization step since $v.d=\infty$ for all $v\in V$
after the initialization step. The inductive hypothesis is $x.d\geq\delta\left(s,x\right)$
for $x\in V$ after the previous relaxation step. During the relaxation
of $\left(u,v\right)$ only $v.d$ might change, and if it does then
it's set to $u.d+w\left(\left(u,v\right)\right)$. But $u.d\geq\delta\left(s,u\right)$
so 
\[
v.d\geq\delta\left(s,u\right)+w\left(\left(u,v\right)\right)\geq\delta\left(s,v\right)
\]
by the triangle inequality. 

Note that once $v.d=\delta\left(s,v\right)$ it cannot decrease further
by the above inequality and it cannot increase because relaxation
only decreases $v.d$. \end{proof}
\begin{cor}
\textup{If there is not path from $s$ to $v$ then $\delta\left(s,v\right)=v.d=\infty$
always.}\end{cor}
\begin{lem}
\textup{\label{lem:Immediately-after-relaxing}Immediately after relaxing
edge $\left(u,v\right)$ by executing Relax$\left(u,v,w\right)$,
$v.d\leq u.d+w\left(\left(u,v\right)\right)$. Basically after at
least one relaxation $v.d$ will never be larger than the distance
to $u$ plus the distance from }\textup{\uline{$u$}}\textup{ to
$v$.}\end{lem}
\begin{proof}
If $v.d>u.d+w\left(\left(u,v\right)\right)$ before relaxation then
$v.d=u.d+w\left(\left(u,v\right)\right)$ after relaxation. Otherwise
if $v.d\leq u.d+w\left(\left(u,v\right)\right)$ before relaxation
then Relax$\left(u,v,w\right)$ is nilpotent and so $v.d\leq u.d+w\left(\left(u,v\right)\right)$
remains.\end{proof}
\begin{thm*}
\textbf{Convergence property}. \textup{Let $G=\left(V,E\right)$ be
a weighted, directed graph with weight function $w\,:\, E\rightarrow\mathbb{R}$
and source vertex $s$, and $s\rightsquigarrow u\rightarrow v$ be
a shortest path for some vertices $u,v\in V$. Suppose $G$ is initialized
using Initialize-Single-Source$\left(G,s\right)$ and then a sequence
of relaxation steps is performed. If $u.d=\delta\left(s,u\right)$
at any time prior calling Relax$\left(u,v,w\right)$ then $v.d=\delta\left(s,v\right)$
at all times afterwards.}\end{thm*}
\begin{proof}
By the upper-bound property if $u.d=\delta\left(s,u\right)$ prior
to relaxing $\left(u,v\right)$ then it continues to hold thereafter.
Hence after relaxing $\left(u,v\right)$ we have $v.d\leq u.d+w\left(\left(u,v\right)\right)$
by Lemma \ref{lem:Immediately-after-relaxing} and 
\begin{eqnarray*}
v.d & \leq & u.d+w\left(\left(u,v\right)\right)\\
 & = & \delta\left(s,u\right)+w\left(\left(u,v\right)\right)\\
 & = & \delta\left(s,v\right)
\end{eqnarray*}
by subpaths of shortest paths are shortest paths%
\footnote{Recall that by assumption $s\rightsquigarrow u\rightarrow v$ is a
shortest path from $s$ to $v$.%
}. Again by Lemma \ref{lem:Immediately-after-relaxing} $v.d\geq\delta\left(s,v\right)$
and so $v.d=\delta\left(s,v\right)$, and the inequality persists
thereafter. \end{proof}
\begin{thm*}
\textbf{Path-relaxation property}. \textup{Let $G=\left(V,E\right)$
be a weighted, directed graph with weight function $w\,:\, E\rightarrow\mathbb{R}$
and source vertex $s$. Consider any shortest path $p=\left\langle v_{0}=s,\dots,v_{k}=v\right\rangle $.
Suppose $G$ is initialized using Initialize-Single-Source$\left(G,s\right)$
and then a sequence of relaxation steps is performed that includes
relaxing $\left(v_{0},v_{1}\right),\dots,\left(v_{1},v_{2}\right),\dots,\left(v_{k-1},v_{k}\right)$
(i.e. the edges $\left(v_{i},v_{j}\right)$ are relaxed in order).
Then $v.d=\delta\left(s,v\right)$ after the relaxations and persists.}\end{thm*}
\begin{proof}
The proof is by induction on the $i$th edge in $p$ to be relaxed.
For the base case $v_{0}.d=s.d=0=\delta\left(s,s\right)$. By the
upper-bound property $s.d$ never changes.

For the inductive step, assume that $v_{i-1}.d=\delta\left(s,v_{i-1}\right)$.
By the convergence property, after relaxing $\left(v_{i-1},v_{i}\right)$
it's the case that $v_{i}.d=\delta\left(s,v_{i}\right)$.\end{proof}
\begin{thm}
\textbf{Shortest-paths trees}. \textup{Let $G=\left(V,E\right)$ be
a weighted, directed graph with weight function $w\,:\, E\rightarrow\mathbb{R}$
and source vertex $s$ and assume $G$ contains no negative-weight
cycles reachable from $s$. Then after initialization the predecessor
subgraph $G_{\pi}$ forms a rooted tree with root $s$, and any sequence
of relaxation steps on edges of $G$ maintain this property as an
invariant.}
\end{thm}
Summary:
\begin{itemize}
\item Triange inequality: $\delta\left(s,v\right)\leq\delta\left(s,u\right)+w\left(\left(u,v\right)\right)$
\item Upper-bound property: $v.d\geq\delta\left(s,v\right)$
\item Convergence property: $s\rightsquigarrow u\rightarrow v$ a shortest
path and $u.d=\delta\left(s,u\right)$ prior to relaxing $\left(u,v\right)$
then $v.d=\delta\left(s,v\right)$ afterward
\item Path-relaxation property: $p=\left\langle s=v_{0},\dots,v_{k}=v\right\rangle $
a shortest path and edges are relaxing in order then $v.d=\delta\left(s,v\right)$.
\item Predecessor-subgraph property: once $v.d=\delta\left(s,v\right)$
for all $v\in V$, the predecessor subgraph is a shortest-paths tree
rooted at $s$.\end{itemize}
\begin{xca}
[24.4-9] This is a dumb problem but for completeness I'm writing
it up%
\footnote{Answer stolen from https://dspace.mit.edu/bitstream/handle/1721.1/37150/6-046JFall-2004/NR/rdonlyres/Electrical-Engineering-and-Computer-Science/6-046JFall-2004/EC0B786D-A562-4952-B52A-7A4DCEB60908/0/ps6sol.pdf%
} . First of all $\max\left\{ x_{i}\right\} =0$. Why? Let $p=\left\langle s,v_{1},\dots,v_{k}\right\rangle $
be the shortest path from the source to any vertex $v_{k}$ returned
by Bellman-Ford. By optimal substructure $p'=\left\langle s,v_{1}\right\rangle $
is the shortest path from the source to $v_{1}$. But by construction
that edge has weight 0 and therefore $x_{1}=0$. Finally since the
$x_{i}\leq0$ it's the case that $\max\left\{ x_{i}\right\} =0$.
Now all that remains is to show that Bellman-Ford maximizes $\min\left\{ x_{i}\right\} $. 

Let $x_{k}=\min\left\{ x_{i}\right\} $ and consider $p=\left\langle s,v_{1},\dots,v_{k}\right\rangle $.
The weight of the path is just the sum of the constraints along the
path
\begin{eqnarray*}
x_{1}-x_{0} & \leq & 0\\
x_{2}-x_{1} & \leq & C_{1,2}\\
 & \vdots\\
x_{k}-x_{k-1} & \leq & C_{k-1,k}
\end{eqnarray*}
Summing we get $x_{k}\leq\sum_{i=1}^{k-1}C_{i,i+1}=w\left(p\right)$
the distance from source to $x_{k}$. Therefore any solution $x_{k}$
that satisfies the constraints cannot be greater than $w\left(p\right)$,
the shortest distance from source to $v_{k}$. Since Bellman-Ford
sets $x_{k}=w\left(p\right)$ it does indeed maximize $x_{k}$. I'm
guessing that this same reasoning can be applied to argue that Bellman-Ford
maximizes $\sum x_{k}$.\end{xca}
\begin{problem}
[24-1]Yen's improvement to Bellman-Ford
\begin{enumerate}
\item [(a)]Towards a contradiction assume that $G_{f}$ has a cycle. Then
there exists a sequence of edges 
\[
\left\{ \left(v_{a},v_{b}\right),\left(v_{c},v_{d}\right),\dots,\left(v_{j},v_{a}\right)\right\} \subset E_{f}
\]
but by definition $\left(v_{j},v_{a}\right)\in E_{b}$ not in $E_{f}$.
Similarly for $G_{b}$. Topological sort follows from definition too:
only edges that ``flow'' in the corresponding direction exist in
each of $E_{f},E_{b}$.
\item [(a)]Stolen from%
\footnote{http://11011110.livejournal.com/215330.html%
}. In the first part of any iteration, any path in the shortest path
tree that starts at a finished vertex and consists of edges only in
$E_{f}$ becomes correctly labeled (shortest path in DAG is easy -
just relax in order of topological sort), i.e. all vertices become
finished. In the second part of any iteration, any path in the shortest
path tree that starts at a finished vertex and consists of edges only
in $E_{b}$ becomes correctly labeled. The number of iterations needed
then is the maximum alternations between $G_{f}$ and\textbf{ $G_{b}$},
which is $\left(\left|V\right|+1\right)/2$, since $G_{f}$ and $G_{b}$
partition $G$ in half. Another optimization is you don't need to
relax edges coming out of a vertex that wasn't updated in the previous
iteration. A final optimization is randomizing the linear ordering
to prevent shortest paths that alternate between $E_{f}$ and $E_{b}$,
bringing down the number of iterations to $\left|V\right|/3$ on average.
\end{enumerate}
\end{problem}

\begin{problem}
[24-2]Nesting boxes
\begin{enumerate}
\item [(a)] Well duh. If box A can contain box B and box B can contain
box C then obviously A can contain C.
\item [(b)] Sort both sets of dimensions and if one set is entry-wise prior
then it fits.
\item [(c)] Do the $n^{2}$ comparisons to figure out all $\left(B_{i},B_{j}\right)$,
where $B_{i}$ can contain $B_{j}$. Then do a topological sort on
the graph because it's a DAG (because what would it mean for a cycle
of box containments). Then find the longest path using DP:
\begin{algorithm}[H]
\begin{lstlisting}[language=Python,mathescape=true,numbers=left,tabsize=4]
Longest-Path(G)
T = Top-Sort(G)
for $v\in T$
	$v.d = \max_{\left(u,v\right)\in E}\left\{ u.d+1\right\} $
return $\max_{v\in V}\left\{ v.d\right\} $
\end{lstlisting}
\caption{Longest Path}


\end{algorithm}

\end{enumerate}
\end{problem}

\begin{problem}
[24-3]Arbitrage
\begin{enumerate}
\item [(a)]Take log. Then Bellman-Ford will find negative cycles.
\end{enumerate}
\end{problem}

\begin{problem}
[24-6]Bitonic shortest paths. Sort the edges by weight. Then relax
increasing order, decreasing order, then increasing again and you're
done\end{problem}

\end{document}
